\chapter{Related Work}\label{cha:RelatedWork}


% Different GAN-based architectures have been developed due to the instability in training models using standard GANs, which often result in ridiculous output by the generator network \cite{radford2015unsupervised}. Therefore, researchers have proposed using other GANs based architectures to create synthetic image datasets to augment currently limited available datasets for training purposes. For example,  Gandhi et al. \cite{gandhi2018plant} used deep convolutional GAN (DCGAN \cite{radford2015unsupervised}) for creating synthetic images to augment the limited number of local images available for training disease detection models for plants in India. Likewise, Barth et al. \cite{barth2020optimising} in their experiment used Cycle GAN for creating 10 500 synthetic images of bell pepper for their segmentation task, Zhu et al. \cite{zhu2018data} created synthetic images from the CVPPP 2017 LSC plant dataset using conditional GAN (CGAN).

% ✅✅✅✅✅
% \section{Plant diseases monitoring in Sugar beet}
Precision agriculture uses innovative technologies to optimise agricultural production through the use of site-specific management. Particularly in crop production, this aims at crop-specific nutrient deficiencies and disease monitoring, application of fertiliser or pesticides, prediction of yields, automated counting of crops, remote/automatic control of agricultural vehicles like drones and tractors. The underlining principle behind precision farming is to use innovative technologies to economically optimise agricultural production as well as reduce harmful outputs into the environment by applying only the needed amount of fertiliser or pesticides needed during a plantation season \cite{auernhammer2001precision}. Applying the specific amount of resources needed by individual plants will help reduce the impact of chemical by-products or hazardous materials ending up in the environment and economically improve farmers’ financial costs. Furthermore, in order to achieve environmental protection of waters and soils, there is a need for reduction in the use of fertilisers and pesticides in crop production \cite{otero2005fertiliser}.
However, the most significant limiting factors in precision farming is the interpretation of collected data from sensors \cite{ondoua2017precision} and adequate datasets. Nevertheless, with the current advancement of data manipulation techniques in data science, there is a possibility of addressing the data interpretation shortcomings to enable a more successful implementation of precision farming.

Based on the research results conducted in our previous work \cite{project_work}, there are many approaches to disease detection in plants, especially in sugar beet. However, there are not enough datasets around prompting researchers to use image augmentation techniques like random cropping and flipping to increase the data size and create randomness in the datasets. Likewise, some of the datasets used in the paper reviewed were not publicly accessible. Hence, the approach in this research is to synthetically generate a dataset using a combination of techniques in image processing. While there are no approaches that precisely match the one described in this thesis, the idea of synthetic image generation for model training already exists.
This chapter reviews different approaches documented in research works in synthetically generating datasets for varying reasons like data privacy and increasing datasets.


Ward et al. \cite{ward2018deep}proposed a method of meeting the large amount of data required to train models using state-of-the-art machine learning approaches for leaf instance segmentation tasks. The proposed framework aims to use the synthetically generated images of the $Arabidopsis$ plant in their research to augment existing natural plant datasets. Their synthetic image dataset generation begins with manually tracing a randomly chosen $Arabidopsis$ leaf image in Blender to produce a 3D mesh of an inspiration leaf based on the original leaf image. Then, more 3D leaf images are created by randomly scaling to model leaves of different shapes and sizes. Next, the synthetically generated plant leaves are circularly arranged close to each other at a similar height. Finally, random background, camera and lightning are added to the generated plant leaves, which are then rendered together as a 2D synthetic image and corresponding segmentation mask. Their proposed framework's success is evidenced by its average accuracy of 90\% in leaf segmentation.

Da Silva et al. \cite{da2019estimating} proposed three methods for generating synthetic defoliation images used in training CNN-based models to estimate soybean leaf defoliation. Their proposed methods remove leaf-belonging pixels in different ways to simulate actual defoliation in a pre-processed image and returns a new defoliated leaf image along with its level of defoliation. 
The first method simulates defoliation using random sizes of polygons formed in random pixels in the leaf area. The second method makes circles with random radii to remove leaf-belonging pixels in the leaf region. Then, secondary circles with different radii are generated in the circumference of the main circle. The third method is similar to the second method except that the secondary circles are generated within the main circle. They were able to generate 10,000 synthetic defoliation images with each method. Their models were trained only with the synthetically generated images and evaluated using natural images. Their experiments produced an impressive result in estimating soybean leaf defoliation despite being trained using synthetically generated datasets.



Datasets are also synthetically generated in domains other than precision farming. For example, Bj{\"o}rklund et al. \cite{bjorklund2019robust}proposed a framework for generating challenging synthetic license plate images to avoid collecting and annotating the thousands of images required to train a CNN model. 
Likewise, Silvano et al. \cite{silvano2021synthetic} described a combination of techniques to generate large random synthetic images of license plates to supplement their small volumes of available authentic images for training deep learning-based automated license plate recognition systems.

In the field of biomedical image analysis, Svoboda and Ulman \cite{svoboda2012generation} generated synthetic static and time-lapse image sequences of fully 3D fluorescence microscopy images showing the motion of objects of various sizes. Han et al. \cite{han2018gan} generated synthetic multi-sequence brain Magnetic Resonance (MR) images using Generative Adversarial Networks (GANs). Prokopenko et al. \cite{prokopenko2019unpaired} investigated approaches for generating synthetic Computed Tomography (CT) images from actual Magnetic Resonance Imaging (MRI) data using GAN to enable single-modality radiotherapy planning in clinical oncology.


